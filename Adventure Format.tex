\includegraphics[width=3.32707in,height=1.71111in]{media/image1.jpg}

Adventure Title

Lorem ipsum dolor sit amet, putent molestie deserunt vim no, ex ferri
quando sed. His eu case eligendi, graeco iuvaret suscipit mel an. Has eu
consul prompta sanctus, pro nobis volumus at. Sed partiendo adipiscing
eu. No vocibus noluisse sit, ei audiam ancillae apeirian duo, vel
offendit adipiscing ad. Causae invenire instructior ex est Vix tractatos
reprimique in. Usu virtute scaevola ex, equidem repudiare democritum vis
te, ius in populo utamur detracto.

A Xxxx-Hour Adventure for \#\#xx-\#\#xx Level Characters

Insert Campaign Glyph here

AUTHOR NAME

Author

\textbf{Adventure Code:} DDXX\#\#-\#\#

\textbf{Optimized For:} APL \#\#

\textbf{Version:} XXXXXXX

\textbf{Development and Editing:} Claire Hoffman, Travis Woodall

\textbf{Organized Play:} Chris Lindsay

\textbf{D\&D Adventurers League Wizards Team:} Adam Lee, Chris Lindsay,
Mike Mearls, Matt Sernett

\textbf{D\&D Adventurers League Administrators:} Robert Adducci, Bill
Benham, Travis Woodall, Claire Hoffman, Greg Marks,\\
Alan Patrick

DUNGEONS \& DRAGONS, D\&D, Wizards of the Coast, Forgotten Realms, the
dragon ampersand, \emph{Player's Handbook, Monster Manual, Dungeon
Master's Guide,} D\&D Adventurers League, all other Wizards of the Coast
product names, and their respective logos are trademarks of Wizards of
the Coast in the USA and other countries. All characters and their
distinctive likenesses are property of Wizards of the Coast. This
material is protected under the copyright laws of the United States of
America. Any reproduction or unauthorized use of the material or artwork
contained herein is prohibited without the express written permission of
Wizards of the Coast.

©2017 Wizards of the Coast LLC, PO Box 707, Renton, WA 98057-0707, USA.
Manufactured by Hasbro SA, Rue Emile-Boéchat 31, 2800 Delémont, CH.
Represented by Hasbro Europe, 4 The Square, Stockley Park, Uxbridge,
Middlesex, UB11 1ET, UK.

\section{}\label{section}

\section{\texorpdfstring{\\
}{ }}\label{section-1}

\section{Introduction}\label{introduction}

\protect\hypertarget{h.1fob9te}{}{}Welcome to \emph{Adventure
Name}\textbf{, a} D\&D Adventurers League™ adventure, part of the
official D\&D Adventurers League\textbf{™} organized play system and the
\emph{Storyline Name\textbf{™}} storyline season.

Insert two to three sentence paragraph describing physical location of
adventure.

This adventure is designed for \textbf{three to seven \#\#xx-\#\#xx
level characters} and is optimized for \textbf{five characters with an
average party level (APL) of \#\#}. Characters outside this level range
cannot participate in this adventure.

\subsection{Adjusting This Adventure}\label{adjusting-this-adventure}

This adventure provides suggestions in making adjustments for smaller or
larger groups, characters of higher or lower levels, and characters that
are otherwise a bit more powerful than the adventure is optimized for.
You're not bound to these adjustments; they're here for your
convenience.

To figure out whether you should consider adjusting the adventure, add
up the total levels of all the characters and divide the total by the
number of characters (rounding .5 or greater up; .4 or less down). This
is the group's APL. To approximate the \textbf{party strength} for the
adventure, consult the following table.

Determining Party Strength

\textbf{Party Composition Party Strength}

3-4 characters, APL less than Very weak

3-4 characters, APL equivalent Weak

3-4 characters, APL greater than Average

5 characters, APL less than Weak

5 characters, APL equivalent Average

5 characters, APL greater than Strong

6-7 characters, APL less than Average

6-7 characters, APL equivalent Strong

6-7 characters, APL greater than Very strong

Some encounters may include a sidebar that offers suggestions for
certain party strengths. If a particular recommendation is not offered
or appropriate for your group, you don't have to make adjustments.

\subsection{\texorpdfstring{\\
Before Play at the
Table}{ Before Play at the Table}}\label{before-play-at-the-table}

Before you start play, consider the following:

\begin{itemize}
\item
  Read through the adventure, taking notes of anything you'd like to
  highlight or remind yourself of while running the adventure, such as a
  way you'd like to portray an NPC or a tactic you'd like to use in a
  combat. Familiarize yourself with the adventure's appendices and
  handouts.
\item
  Gather any resources you'd like to use to aid you in running this
  adventure-\/-such as notecards, a DM screen, miniatures, and
  battlemaps.
\item
  Ask the players to provide you with relevant character information,
  such as name, race, class, and level; passive Wisdom (Perception), and
  anything specified as notable by the adventure (such as backgrounds,
  traits, flaws, etc.)
\end{itemize}

\subsection{Playing the Dungeon
Master}\label{playing-the-dungeon-master}

You have the most important role---facilitating the enjoyment of the
game for the players. You provide the narrative and bring the words on
these pages to life.

To facilitate this, keep in mind the following:

\emph{\textbf{You're Empowered.}} Make decisions about how the group
interacts with the adventure; adjusting or improvising is encouraged, so
long as you maintain the adventure's spirit. This doesn't allow you to
implement house rules or change those of the Adventurers League,
however; they should be consistent in this regard.

\emph{\textbf{Challenge Your Players.}} Gauge the experience level of
your \textbf{players} (not the characters), try to feel out (or ask)
what they like in a game, and attempt to deliver the experience they're
after. Everyone should have the opportunity to shine.

\emph{\textbf{Keep the Adventure Moving.}} When the game starts to get
bogged down, feel free to provide hints and clues to your players so
they can attempt to solve puzzles, engage in combat, and roleplay
interactions without getting too frustrated over a lack of information.
This gives players ``little victories'' for figuring out good choices
from clues. Watch for stalling---play loses momentum when this happens.
At the same time, make sure that the players don't finish too early;
provide them with a full play experience.

\section{\texorpdfstring{\\
}{ }}\label{section-2}

\section{Adventure Primer}\label{adventure-primer}

This section provides the adventure's background, a list of prominent
NPCs, an overview of the adventure in play, and hooks that you can use
to introduce your players's characters to the action.

Ideally, this section should consist of a maximum of \textbf{one}
page---thorough, but not excessively so. It must start after a page
break.

\subsection{Adventure Background}\label{adventure-background}

This should be in depth, but not excessive; three-to-four paragraphs of
three-to-five sentences each.

Provide a reasonably complete background for the adventure in this
section. The background is the story that takes place before the player
characters come on the scene, but bits and pieces of it are revealed to
the players during the course of the adventure, which allows them to
make some sense of what is going on.

There's a fine line between two little background and too much
exposition. Try to stay on the side of the line that doesn't bloat the
background with backstory that isn't ultimately important.

Location and NPC Summary

The following NPCs and locations feature prominently in this adventure.
Each NPC and location should have an entry with a phonetic pronunciation
and one-three sentences describing them.

\emph{\textbf{Lorem Ipsum (LOW-rum IP-sum).}} Ut enim ad minim veniam,
quis nostrud exercitation ullamco laboris nisi ut aliquip ex ea commodo
consequat.

\emph{\textbf{Lorem Ipsum (LOW-rum IP-sum).}} Ut enim ad minim veniam,
quis nostrud exercitation ullamco laboris nisi ut aliquip ex ea commodo
consequat.

\subsection{Adventure Overview}\label{adventure-overview}

This section provides the DM a bulleted overview of how the adventure is
most likely to play out. If the adventure isn't linear, discuss the
various alternative paths the adventure might take. This section should
call out any special twists that would otherwise be buried in encounters
or other later parts of the adventure, so the DM isn't surprised later.
Each of these entries should consist of 1-3 sentences. Be concise.

It's useful to write the final draft of this section last, after the
rest of the adventure has solidified.

The adventure is broken down into XXX parts:

\emph{\textbf{Part 1.}} Lorem ipsum dolor sit amet, consectetur
adipisicing elit, sed do eiusmod tempor incididunt ut labore et dolore
magna aliqua.

\emph{\textbf{Part2.}} Lorem ipsum dolor sit amet, consectetur
adipisicing elit, sed do eiusmod tempor incididunt ut labore et dolore
magna aliqua.

\emph{\textbf{Part3.}} Lorem ipsum dolor sit amet, consectetur
adipisicing elit, sed do eiusmod tempor incididunt ut labore et dolore
magna aliqua.

\subsection{Adventure Hooks}\label{adventure-hooks}

Here, provide one-to-two paragraphs of two-to-three sentence each that
explain *why* the characters might be interested in participating in the
adventure.

From here, you can should provide a few more, more specific adventure
hooks tied to aspects of the characters, such as backgrounds,
alignments, story awards, and---if the adventure's concept called for
it---faction assignments and secret missions.

\emph{\textbf{Story Hook.}} Lorem ipsum dolor sit amet, consectetur
adipisicing elit, sed do eiusmod tempor incididunt ut labore et dolore
magna aliqua. Ut enim ad minim veniam, quis nostrud exercitation ullamco
laboris nisi ut aliquip ex ea commodo consequat.

\emph{\textbf{Faction (Faction Assignment).}} Lorem ipsum dolor sit
amet, consectetur adipisicing elit, sed do eiusmod tempor incididunt ut
labore et dolore magna aliqua. Ut enim ad minim veniam, quis nostrud
exercitation ullamco laboris nisi ut aliquip ex ea commodo consequat.

\emph{\textbf{Faction (Secret Mission).}} Lorem ipsum dolor sit amet,
consectetur adipisicing elit, sed do eiusmod tempor incididunt ut labore
et dolore magna aliqua. Ut enim ad minim veniam, quis nostrud
exercitation ullamco laboris nisi ut aliquip ex ea commodo consequat.

Secret Mission

Adventures with a secret mission have an encounter that replaces an
existing encounter or makes an existing encounter more difficult.

This sidebar should provide some information for the DM that explains
the nature of the secret mission and how it modifies the adventure.

\section{Part \#\#. General Area or Section
Name}\label{part-.-general-area-or-section-name}

\emph{\textbf{Estimated Duration:}} \#\# minutes

Be very careful in estimating the time needed to complete a part of the
adventure. Underestimating it means that DMs may be forced to cut
encounters short. While overselling it may leave a group finishing an
adventure an hour early.

Provide a brief overview of what the characters should expect to
encounter during this part of the adventure. Parts are numbered
numerically (i.e., 1, 2, 3, etc.).

It should also house the general features of the area that this part
takes place in.

General Features

Lorem has the following general features. These sidebars should be added
whenever a given location's general features should be made known to the
characters. Each entry should be concise and flavorful.

\emph{\textbf{Terrain.}} Lorem ipsum dolor sit amet, consectetur

\emph{\textbf{Weather.}} Lorem ipsum dolor sit amet, consectetur

\emph{\textbf{Light.}} Lorem ipsum dolor sit amet, consectetur
adipisicing

\emph{\textbf{Smells and Sounds.}} Lorem ipsum dolor sit amet,
consectetur adipisicing elit

\subsection{XX. Location, Event or Other
Subsection}\label{xx.-location-event-or-other-subsection}

If the subsection is a location on a map, then the location number
precedes its name. These subsections are numbered alphabetically (i.e.,
A, B, C, etc.)

Boxed text can sit anywhere under an H1, H2, or H3, but it is most
common under an H2 describing a location. It is usually near the
beginning to help set a scene.

It should be no more than three paragraphs of one-to-three sentences
each. Excessive boxed text slows the adventures to a crawl.

Within the running text, \textbf{names and numbers of creatures} or
other significant game objects such as \textbf{traps} or \textbf{secret
doors} are in bold. Use boldface judiciously. With the exception of
creature names, it usually applies to important objects that are keyed
on the map. Bold creature names should match the names that appear in
the statistics.

\subsubsection{Developments, Treasure,
Etc.}\label{developments-treasure-etc.}

H3 headings are most commonly ``Developments'' or ``Treasure,'' but
might also feature complex traps or terrain, notable NPCs, etc.

\emph{\textbf{Inline Subhead Name.}} The inline subhead sits under an H3
and can be used to organize the presented information.

\begin{itemize}
\item
  Core Bulleted. This style is used to organize lists of information,
  such as bits of information an NPC provides and other things that
  characters might have the opportunity to learn.
\end{itemize}

\subsubsection{\#\#xx Sub-location}\label{xx-sub-location}

Heading 3 is also used if a location is divided into multiple smaller
sections. For example, ``14a. Abandoned Jail Cell,'' ``14b. Occupied
Jail Cell.'' This subdivision should only be used if the sub-locations
have enough detail to merit individual descriptions.

\subsubsection{Tricks of the Trade}\label{tricks-of-the-trade}

For each encounter, include tips and tricks to assist the new Dungeon
Master with managing the encounter, as well as helpful hints for keeping
the players on track. Each monster, NPC, or environmental aspect should
have its own, separate entry.

\emph{\textbf{Combat Encounters.}} Include ideas on cool strategies the
NPCs/Monsters might use, or teamwork they employ. If the monsters are
not particularly bright, this can include direction on a lack of
planning or tactics.

\emph{\textbf{Interaction Encounter.}} Include notes on the motivation,
demeanor, and disposition of the NPC(s). Also, include how they respond
to the use of Deception, Intimidation, and/or Persuasion, and what
information they might have to offer.

\emph{\textbf{Exploration Encounter.}} Include notes on how making
specific choices in the use of Investigation or Perception might be
helpful or not; on what the DM might do if the adventurers take
precautions or if they charge right into the area.

Regardless of the type of encounter, this subheading should include a
note that indicates if the DM would like to run it differently they can,
and that this is but ONE possible way the encounter can roll out.

\subsection{Area E. Nesting Area}\label{area-e.-nesting-area}

Due to its proximity to Sik'garuk's lair, this nesting area is reserved
for those in his favor---typically the strongest, smartest, or most
comely kobolds of the tribe. Read or paraphrase:

General Features

The nesting area has the following general features:

\emph{\textbf{Terrain.}} The floor here is natural stone and is more or
less level. The walls are worked and bear the marks of small,
kobold-sized tools.

\emph{\textbf{Weather.}} The air here is humid and warm. The walls are
damp from the humidity and sport small patches of mold.

\emph{\textbf{Light.}} There is no light here save for that which the
characters brought with them.

\emph{\textbf{Smells and Sounds.}} Rotting straw and meat, mildew, the
tang of acidic air. Scratching of claws on stone, and insects.

Much like the chamber you encountered previously, this area is littered
with piles of straw, leaves, and scraps of cloth---though the pallets
are fewer in number and not quite so moldy.

It would appear that the kobolds that live in this area of the warren
take a little more pride in their surroundings as the area is lacking
the rather rude graffiti found elsewhere in the warren.

Sik'garuk, a \textbf{kobold} \textbf{scale sorcerer,} is here with a
trio of \textbf{kobolds scouts}. If the warren is on alert, the scale
sorcerer is astride his trusty \textbf{giant weasel} mount. The giant
weasel wears leather barding (this increases its AC to 14).

If the scale sorcerer is slain, one of the surviving kobolds flee
towards Sik'garuk's lair (see Part 3, below) to warn the tribe's leader.

Adjusting this Encounter

Here are some suggestions for adjusting this encounter, according to
your group. These are not cumulative.

\begin{itemize}
\item
  \textbf{Very Weak:} The \textbf{scale sorcerer} has no
  2\textsuperscript{nd} level spell slots; remove two \textbf{scouts}
\item
  \textbf{Weak:} Remove two \textbf{scouts}
\item
  \textbf{Strong:} Replace two \textbf{scouts} with \textbf{spies}
\item
  \textbf{Very Strong:} Add a \textbf{dragonshield}
\end{itemize}

\subsubsection{Treasure}\label{treasure}

The kobolds carry a collective treasure of 38 gp worth of coins and
gems.

Additionally, a character that spends 30 minutes searching the nest and
succeeds on a DC 13 Intelligence (Investigation) check discovers that
one of the rocks in the area is actually a large sheet of rock-covered
canvas. Beneath the canvas is a medium-sized rock bearing a sign reading
(in horrible-written Draconic):

``Nothing under here---don't look.''

A character disobeying the sign finds a garnet worth 12 gp in an old,
manky sock hidden beneath the rock.

\subsubsection{Development}\label{development}

If the characters capture Sik'garuk, he feigns ignorance and is
generally difficult to deal with.

Roleplaying Sik'garuk

This spunky kobold is eight-feet tall in spirit, if not in body. He's
heard all these rumors about kobolds being scaredy-cats, and will have
no truck with it. He's mean, headstrong, and cunning---a deadly
combination.

His ruddy scales are covered in large splotches of black and a pair of
vestigial, leathery wings have sprouted from his back---gifts from his
matron, Nightscale.

\emph{\textbf{Quote:}} \emph{``No! You no be here! Is hers! IS HERS!''}

\subsection{Development}\label{development-1}

From here, the characters proceed to Part 3, below. However, the hallway
out of this chamber has a cunningly disguised \textbf{pit}
\textbf{trap}.

Trap. Jump Here!

It's only 10-feet deep, but this \textbf{pit} is lined with sharpened
sticks immersed in a few feet of fetid, murky water. The kobolds avoid
the trap by either jumping over it or by using the secondary tunnels.

\emph{\textbf{Detection and Disabling.}} The 5-foot wide pit is covered
with a large piece of canvas painted to resemble the tunnel floor.
Detecting it requires a successful DC 15 Intelligence (Investigation)
check. If successful, the dimensions of the pit can be discerned from
the tunnel floor. It can't be disabled, but it can be jumped over, etc.,
normally.

\emph{\textbf{Trigger.}} Stepping onto the false floor triggers the
trap.

\emph{\textbf{Effect.}} The triggering character must succeed on a DC 13
Dexterity saving throw or fall into the 10-foot deep pit and takes 3
(1d6) bludgeoning damage from the fall and 5 (1d10) piercing damage from
the spikes on the bottom of the pit. As the spikes are submerged in
fetid water, if the triggering creature is reduced to zero hit points,
they slip beneath the surface of the opaque water.

\subsubsection{XP Award}\label{xp-award}

If the characters detect and avoid the pit trap, award each character 50
XP.

\section{\texorpdfstring{\\
}{ }}\label{section-3}

\section{Rewards}\label{rewards}

Make sure players note their rewards on their adventure log sheets. Give
your name and DCI number (if applicable) so players can record who ran
the session.

\subsection{Experience}\label{experience}

Total up all combat experience earned for defeated foes, and divide by
the number of characters present in the combat. For non-combat
experience, the rewards are listed per character.

Combat Awards

\textbf{Name of Foe XP Per Foe}

Lorem Ipsem \#,\#\#\#

Non-Combat Awards

\textbf{Task or Accomplishment XP Per Character}

Lorem Ipsem \#,\#\#\#

The adventures minimum and maximum XP awards are located on the
Adventure Rewards document. Note that DDAL adventures award a maximum of
the ``target XP'' award. Only epics may grant the ``maximum'' amount.

The \textbf{minimum} total award for each character participating in
this adventure is XXXX \textbf{experience points}.

The \textbf{maximum} total award for each character participating in
this adventure is XXXX \textbf{experience points.}

\subsection{Treasure}\label{treasure-1}

The characters receive the following treasure, divided up amongst the
party. Treasure is divided as evenly as possible. Gold piece values
listed for sellable gear are calculated at their selling price, not
their purchase price.

Treasure Awards

\textbf{Item Name GP Value}

Lorem Ipsem 50

\emph{\textbf{Consumable magic items}} should be divided up however the
group sees fit. If more than one character is interested in a specific
consumable magic item, the DM can determine who gets it randomly should
the group be unable to decide.

\emph{\textbf{Permanent magic items}} are divided according to a system
detailed in the \emph{D\&D Adventurers League Dungeon Master's Guide}.

\subsubsection{Gauntlets of Lorem Ipsem}\label{gauntlets-of-lorem-ipsem}

Wondrous Item, uncommon

Lorem ipsum dolor sit amet, consectetur adipisicing elit, sed do eiusmod
tempor incididunt ut labore et dolore magna aliqua. Ut enim ad minim
veniam, quis nostrud exercitation ullamco laboris nisi ut aliquip ex ea
commodo consequat. This item can be found in \textbf{Player Handout
XXX}\emph{.}

\subsubsection{Potion of Lorem Ipsem}\label{potion-of-lorem-ipsem}

Potion, common

This item can be found in the \emph{Player's Handbook.}

\subsubsection{Scroll of Lorem Ipsem}\label{scroll-of-lorem-ipsem}

Potion, uncommon

This item can be found in the \emph{Dungeon Master's Guide.}

\subsection{Downtime Activities}\label{downtime-activities}

During the course of this adventure, the characters may earn access to
the following downtime activity:

\emph{\textbf{Inline Subhead.}} Lorem ipsum dolor sit amet, consectetur
adipisicing elit, sed do eiusmod tempor incididunt ut labore et dolore
magna aliqua. Ut enim ad minim veniam, quis nostrud exercitation ullamco
laboris nisi ut aliquip ex ea commodo consequat. More information can be
found in \textbf{Player Handout XX}.

\subsection{Story Awards}\label{story-awards}

During the course of this adventure, the characters may earn the
following story award:

\emph{\textbf{Inline Subhead.}} Lorem ipsum dolor sit amet, consectetur
adipisicing elit, sed do eiusmod tempor incididunt ut labore et dolore
magna aliqua. Ut enim ad minim veniam, quis nostrud exercitation ullamco
laboris nisi ut aliquip ex ea commodo consequat. More information can be
found in \textbf{Player Handout XX}.

\subsection{Renown}\label{renown}

Each character receives \textbf{one renown} at the conclusion of this
adventure.

\textbf{Members of \textless{}FACTION\textgreater{}} that lorem ipsum
dolor sit amet, consectetur adipisicing elit, sed do eiusmod tempor
incididunt ut labore dolore magna earn \textbf{one additional renown
point}.

\textbf{Members of \textless{}FACTION\textgreater{} (rank 2 or higher)}
that lorem ipsum dolor sit amet, consectetur adipisicing elit, sed do
tempor incididunt ut labore dolore magna earn \textbf{one additional
renown point} and mark the completion of a secret mission on their
adventure logsheet.

\subsection{DM Reward}\label{dm-reward}

In exchange for running this adventure, you earn DM Rewards as described
in the \emph{D\&D Adventurers League Dungeon Master's Guide} (ALDMG).

\section{Appendix. Dramatis Personae}\label{appendix.-dramatis-personae}

The following NPCs are featured prominently in this adventure:

\emph{\textbf{Lorem Ipsum (LOW-rum IP-sum).}} Each NPC should have a
brief, 3-5 sentence summarization. Note that this description is
slightly longer than that of the sidebar in the adventure primer, above.
Additionally, this appendix may also include NPCs that, while not as
important as those listed in the primer, warrant their own entry.

\section{Appendix. Monster/NPC
Statistics}\label{appendix.-monsternpc-statistics}

Yellow Musk Creeper

Medium plant, unaligned

\textbf{Armor Class} 6

\textbf{Hit Points} 60 (11d8 + 11)

\textbf{Speed} 5 ft., climb 5 ft.

\textbf{STR DEX CON INT WIS CHA}

12 (+1) 3 (-4) 12 (+1) 1 (-5) 10 (+0) 3 (-4)

\textbf{Condition Immunities} blinded, deafened, exhaustion, prone

\textbf{Senses} blindsight 30 ft., passive Perception 10

\textbf{Languages} ---

\textbf{Challenge} 2 (450 XP)

\emph{\textbf{False Appearance.}} While the creeper remains motionless,
it is indistinguishable from an ordinary flowering vine.

\emph{\textbf{Regeneration.}} The creeper regains 10 hit points at the
start of Its turn. If the creeper takes fire, necrotic, or radiant
damage, this trait doesn't function at the start of its next turn. The
creeper dies only if it starts its turn with 0 hit points and doesn't
regenerate.

Actions

\emph{\textbf{Touch.} Melee Weapon Attack:} +3 to hit, reach 5 ft., one
creature. \emph{Hit:} 13 (3d8) psychic damage. If the target is a
humanoid that drops to 0 hit points as a result of this damage, it dies
and is implanted with a yellow musk creeper bulb. Unless the bulb is
destroyed, the corpse animates as a yellow musk zombie after being dead
for 24 hours. The bulb is destroyed if the creature is raised from the
dead before it can transform into a yellow musk zombie, or if the corpse
is targeted by a \emph{remove curse} spell or similar magic before it
animates.

\emph{\textbf{Yellow Musk (3/day). }}

musk that targets all humanoids within 30 feet of it. Each

target must succeed on a DC 11 Wisdom saving throw or be

charmed by the creeper for 1 minute. A creature charmed in

this way does nothing on its turn except move as close as it can

to the creeper. A creature charmed by the creeper can repeat

the saving throw at the end of each of its turns, ending the effect

\section{on itself on a success.}\label{on-itself-on-a-success.}

Swarm of Centipedes

Medium swarm of Tiny beasts, unaligned

\textbf{Armor} \textbf{Class} 12 (natural armor)

\textbf{Hit} \textbf{Points} 22 (5d8)

\textbf{Speed} 20 ft., climb 20 ft.

\textbf{STR DEX CON INT WIS CHA}

3 (-4) 13 (+1) 10 (+0) 1 (-5) 7 (-2) 1 (-5)

\textbf{Damage Resistances} bludgeoning, piercing, slashing

\textbf{Condition Immunities} charmed, frightened, paralyzed, petrified,
prone, restrained, stunned

\textbf{Senses} blindsight 10 ft., passive Perception 8

\textbf{Languages} -\/-

\textbf{Challenge} 1/2 (100 XP)

\emph{\textbf{Swarm.}} The swarm can occupy another creature's space and
vice versa, and the swarm can move through any opening large enough for
a Tiny insect. The swarm can't regain hit points or gain temporary hit
points.

\emph{\textbf{Paralyzing Bites.}} A creature reduced to 0 hit points by
a swarm of centipedes is stable but poisoned for 1 hour, even after
regaining hit points, and paralyzed while poisoned in this way.

Actions

\emph{\textbf{Bite.}} \emph{Melee Weapon Attack:} +3 to hit, reach 0
ft., one target in the swarm's space. \emph{Hit:} 10 (4d4) piercing
damage or 5 (2d4) piercing damage if the swarm has half of its hit
points or fewer.

\section{\texorpdfstring{\\
}{ }}\label{section-4}

\section{\texorpdfstring{Appendix. Map\\
}{Appendix. Map }}\label{appendix.-map}

\section{\texorpdfstring{Appendix. DM Handout\\
}{Appendix. DM Handout }}\label{appendix.-dm-handout}

\section{\texorpdfstring{Player Handout \#.\\
}{Player Handout \#. }}\label{player-handout-.}

\section{Player Handout \#. Downtime
Activity}\label{player-handout-.-downtime-activity}

During the course of this adventure, the characters may earn access to
the following downtime activity. If you are printing these out for your
characters, print as many as you may need to ensure that any eligible
character receives a copy:

\subsection{Downtime Activity}\label{downtime-activity}

Lorem ipsum dolor sit amet, consectetur adipisicing elit, sed do eiusmod
tempor incididunt ut labore et dolore magna aliqua. Ut enim ad minim
veniam, quis nostrud exercitation ullamco laboris nisi ut aliquip ex ea
commodo consequat.

\section{Player Handout \#. Story
Award}\label{player-handout-.-story-award}

During the course of this adventure, the characters may earn the
following story award. If you are printing these out for your
characters, print as many as you may need to ensure that any eligible
character receives a copy:

\subsection{Story Award}\label{story-award}

Lorem ipsum dolor sit amet, consectetur adipisicing elit, sed do eiusmod
tempor incididunt ut labore et dolore magna aliqua. Ut enim ad minim
veniam, quis nostrud exercitation ullamco laboris nisi ut aliquip ex ea
commodo consequat.

\section{Player Handout \#. Magic
Item}\label{player-handout-.-magic-item}

During the course of this adventure, the characters may find the
following permanent magic item:

\subsection{Gauntlets of Lorem Ipsem}\label{gauntlets-of-lorem-ipsem-1}

Wondrous Item, uncommon

The first portion of this entry should be the description of the item
found in the Dungeon Master's Guide (or whatever resource the item is
officially described in).

The second portion should be whatever cosmetic description or additional
properties that the item has by virtue of the adventure. This is
typically copied directly from the Adventure Concept.

This item can be found in the \emph{Name of Resource}.
